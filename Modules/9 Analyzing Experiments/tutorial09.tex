\documentclass[12pt]{article}
\usepackage{../tutorialsty}
\usepackage[urlcolor=blue]{hyperref}
\urlstyle{sf}
%\def\UrlBreaks{\do/\do.\do-}



% For algorithms
%\usepackage{algorithm,algorithmic}

% Bibliography
%\usepackage[style=numeric, backend=bibtex8, doi=false, url=false, isbn=false, eprint=false]{biblatex}
%\addbibresounce{./example_bibfile.bib}

\begin{document}
\title{Module 9}
\author{Ryo Kimura}
\date{}
\maketitle

\section{Analyzing Experiments}
\subsection{Analyzing Results}
\subsubsection{No Free Lunch Theorem}
Don't just report statistics about which algorithm performed best, but give some insight as to \emph{why} certain algorithms do better than others.

Talk about think-cell addin. (See CMU Software)

\subsubsection{Metrics}
Tables, have pandas generate them for you, determine what to show (comprehensive and unwieldy, or incomplete but highlight key metrics?)
\subsubsection{software}
pandas (depends on numpy)

\subsection{Visualizing Results}
\subsubsection{Plots for Algorithm Properties}
Trajectory plots: (plot the intermediate solutions given by the iterative algorithm (e.g., gradient descent)): does the algorithm exhibit ``zig-zagging'' behavior? does it get ``stuck'' in a sub-optimal region?
Convergence plots (plot the best objective value/lower bound/upper bound found during the solution procedure): does the algorithm find a good/optimal solution/bound quickly?

\subsubsection{Plots for Algorithm Comparisons}
Comparison Plots (diagonal in the middle, vertical axis is new/proposed method, horizontal axis is standard/benchmark method; lower right is better, under diagonal is good; scatter plot of points for each instance, distiguish points by color/shape/size)
CDF of Performance (Think of performance as a probability distribution, with each instance $i$ being associated with a random variable $P(I)$ representing the performance of the algorithm on instance $i$ (e.g., solution time) vertical axis is number of instances solved, horizontal axis is time)

% Ref: Dolan and More, Benchmarking Optimization Software with Performance Profiles
% Ref: Beiranvand, Hare, Lucet: Best practices for comparing optimization algorithms
\subsubsection{Python Graphing Packages}
\begin{itemize}
    \item \textbf{Matplotlib} - fine-grained control over plotting, can be unwieldy for more complex plots
    \item \textbf{Seaborn} - built on top of Matplotlib, extremely popular, versatile FacetGrid function, lots of dedicated functions as well
    \item \textbf{plotly} - built for interactive plots, exporting to static image requires additional dependencies
    \item \textbf{Altair} - built on top of Vega Lite, newest package, delightfully simple syntax, missing some features here and there (e.g., no boxplots)
    \item \textbf{plotnine} - for the ex-R programmers who miss ggplot2
\end{itemize}
\end{document}
